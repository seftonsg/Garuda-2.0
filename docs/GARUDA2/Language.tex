% Sage Sefton
% G2
% Abstract Due Apr 3


%Preamble%
  \documentclass[12pt, letterpaper]{article}
  \usepackage{pdfpages} %for including PDFs. see Graphicx for more info
  \usepackage{amsmath} %for \mvert and other relation symbol
  \usepackage{amssymb} %for semantic delta=
  \usepackage{amsthm} %for \begin{proof} and related sections
  \usepackage{stmaryrd} %for semantics [[ and ]]
  \usepackage{dsfont} %for fancy set letters
  \usepackage{fancyhdr} %fancy headers
  \usepackage{color} %for things that need attention
  \usepackage[numbers]{natbib}
  \usepackage{xspace}
  %\usepackage{array} %Include if {tabular}{>{$}c<{$}} type arrays are required
  \let\emptyset\varnothing
  \pagestyle{fancy}
  \lhead{Sage Sefton}
  \rhead{Page \thepage}

%Custom Commands%
  \newcommand\interp[1]{\llbracket #1 \rrbracket}
  \def \sysname {\textsc{GARUDA 2.0}\xspace}
  \def \oldname {\textsc{GARUDA}\xspace}

\begin{document}

%%%%%%%%%%%%%%%%%%%%%%%%
%%% BEGIN TITLE PAGE %%%
%%%%%%%%%%%%%%%%%%%%%%%%
  \title{\sysname}
  \date{2020\\ Fall}

  \author{
    Sage Sefton \\
    ss557415@ohio.edu \\
    Ohio University \\
    \and
    Avinash Karanth  \\
    karanth@ohio.edu \\
    Ohio University  \\
  %  \and
  %  Gordon Stewart \\
  %  gstewart@ohio.edu \\
  %  Ohio University
  }

  \maketitle
  \cleardoublepage
%%%%%%%%%%%%%%%%%%%%%%
%%% END TITLE PAGE %%%
%%%%%%%%%%%%%%%%%%%%%%

%%%%%%%%%%%%%%%%%%%%%%%
%%% BEGIN SECTION 1 %%%
%%% BACKGROUND      %%%
%%%%%%%%%%%%%%%%%%%%%%%
  \section{Background}\label{sec:bkgnd}

    \oldname was a system that operated on a two stream model of the "user" or program, and that which executed the instructions.
    \sysname, we extend, or refine based on the point of view, this model to operate between the Execution (EXE) and Memory (MEM) stages of the pipeline.
    We assume a monitor theory shown by the graphic Fig~\ref{fig:bkgnd:newmodel}.
    The monitor appears to operate on a single stream, but the two hanging connections can be considered the two streams of the former model.

    The most obvious change we need to make is to enable encryption and decryption of the EffAddr.
    Our attack vector focuses on a trojan targeting the state register between the EXE and MEM stages.
    They could cache these addresses and try to access that memory later, or they may try to use them as a side channel.
    Either way, allowing an adversary to track our memory accesses is clearly problematic.
    However, when we refer to "encryption", we really mean some reversible obfuscation function.
    We employ the proof assistant, Coq, to prove the EN and DE blocks are exact inverses.
    \oldname is written in Coq for exactly this reason, so requiring a proof that $\forall P, DE (EN P) = P$ should be sufficient.
    % TODO: Check this proof format, it may be wrong



%%%%%%%%%%%%%%%%%%%%%
%%% END SECTION 1 %%%
%%% BACKGROUND    %%%
%%%%%%%%%%%%%%%%%%%%%


%%%%%%%%%%%%%%%%%%%%%%%%
%%% BEGIN SECTION 2  %%%
%%% GARUDA SPECIFICS %%%
%%%%%%%%%%%%%%%%%%%%%%%%
  \section{The Implementation of \sysname}\label{sec:spec}

  %%%%%%%%%%%%%%%%%%%%%%%%%
  %%% BEGIN SECTION 2.1 %%%
  %%% GARUDA SYNTAX     %%%
  %%%%%%%%%%%%%%%%%%%%%%%%%
    \subsection{Syntax}\label{sec:spec:synt}
      \subsubsection{Definitions}\label{sec:spec:synt:defn}
        %Definitions table
        \begin{figure}
          \centering
          \begin{tabular}{l r c l}
            EXE Input Fields  & $f_{Ei}$  & $::=$ & $f_{Ei_{1}} \mid \dots \mid f_{Ei_{k}}$\\
            EXE Output Fields & $f_{Eo}$  & $::=$ & $f_{Eo_{1}} \mid \dots \mid f_{Eo_{k}}$\\
            MEM Input Fields  & $f_{Mi}$  & $::=$ & $f_{Mi_{1}} \mid \dots \mid f_{Mi_{k}}$\\
            Fields            & $f$       & $::=$ & $f_{Ei} \mid f_{Eo} \mid f_{Mi} $ \\
            Inputs            & $i$       & $::=$ & $\{f_{i_{1}} = v_{i_{1}} ,\ \dots\ ,\ f_{i_{k}} = v_{i_{k}}\}$\\
            Outputs           & $o$       & $::=$ & $\{f_{o_{1}} = v_{o_{1}} ,\ \dots\ ,\ f_{o_{k}} = v_{o_{k}}\}$\\
            Obfuscation Fxn   & $\lambda$ & $::=$ & $f \rightarrow f'$
            % Obfuscation Fxn   & $\mathbf{F}$  & $::=$ & $f \rightarrow f'$
            % TODO: There's got to be something clearer than F, maybe capital gamma, lambda?
            % TODO: how do we define state in and state out?
            % TODO: Mention that Inputs and Outputs can generically refer to either EXE or MEM.
            %       Indeed, the Outputs refers to both.
          \end{tabular}
          \caption{Definitions in \sysname.}
          \label{fig:spec:synt:defn}
        \end{figure}

      \subsubsection{Syntax of Predicates}\label{sec:spec:synt:pred}
        %Predicates table
        \begin{figure}
          \centering
          \begin{tabular}{l c r l l}
            Predicates  & $a,b$ & $::=$  & $0$          & \textit{False}    \\
                        &       & $\mid$ & $1$          & \textit{Identity} \\
                        &       & $\mid$ & $f = n$      & \textit{Test}     \\
                        &       & $\mid$ & $a + b$      & \textit{Sum}      \\
                        &       & $\mid$ & $a \cdot b$  & \textit{Product}  \\
                        &       & $\mid$ & $\neg\ a$    & \textit{Negation}
          \end{tabular}
          \caption{Predicates in \sysname act as a boolean algebra.}
          \label{fig:spec:synt:pred}
        \end{figure}

      \subsubsection{Syntax of Policies}\label{sec:spec:synt:pol}
        %Policies table
        \begin{figure}
          \centering
          \begin{tabular}{l c r l l}
            Policies  & $p,q$ & $::=$  & $\mathbf{test}(a)$    & \textit{Test}     \\
                      &       & $\mid$ & $(\lambda_{Encrypt}, 
                                           \lambda_{Decrypt})$ & \textit{Obfuscate} \\
                      &       & $\mid$ & $inj_{Eo}$            & \textit{Injection Execution Output} \\
                      % alternatively: Injection to State Register (inj_{i})
                      &       & $\mid$ & $inj_{Mi}$            & \textit{Injection Memory Input} \\
                      % alternatively: Injection from State Register (inj_{o})
                      &       & $\mid$ & $f \leftarrow n$      & \textit{Update}   \\
                      &       & $\mid$ & $p + q$               & \textit{Choice}   \\
                      &       & $\mid$ & $p \cdot q$           & \textit{Sequential Concatenation} \\
          \end{tabular}
          \caption{Syntax of Policies in \sysname}
          \label{fig:spec:synt:pol}
        \end{figure}
  %%%%%%%%%%%%%%%%%%%%%%%
  %%% END SECTION 2.1 %%%
  %%% GARUDA SYNTAX   %%%
  %%%%%%%%%%%%%%%%%%%%%%%

  %%%%%%%%%%%%%%%%%%%%%%%%%
  %%% BEGIN SECTION 2.2 %%%
  %%% GARUDA SEMANTICS  %%%
  %%%%%%%%%%%%%%%%%%%%%%%%%
    \clearpage
    \subsection{Semantics}\label{sec:spec:sem}
      \subsubsection{Semantics of Predicates}\label{sec:spec:sem:pred}
    
        %Semantics of Predicates
        \begin{figure}
          \centering
          \begin{align*}
            % Theory
            \interp{ \cdot }\ 
              :\ \ &
              \mathbf{Stream}(Eo)\times \mathbf{Stream}(Mi) \rightarrow \\
              & P(\mathbf{Stream}(Eo))\times P(\mathbf{Stream}(Mi)) 
              \\
            % Falsity
            \interp{ 0 }(-, -)
              \triangleq\ &
              (\emptyset , \emptyset)
              \\ % or null-stream?
            % Identity
            \interp{ 1 }(\mathit{os}, \mathit{is})
              \triangleq\ &
              (\{\mathit{os}\},\{\mathit{is}\})
              \\
            % Test
            \interp{ f=n }(\mathit{os}, \mathit{is})
              \triangleq\ &
              (\mathsf{filter}\ (f=n)\ \{\mathit{os}\},\
               \mathsf{filter}\ (f=n)\ \{\mathit{is}\}) 
              \\
            % Sum
            \interp{ a + b }(\mathit{os}, \mathit{is})
              \triangleq\ &
              \interp { a }(\mathit{os}, \mathit{is})\cup
              \interp { b }(\mathit{os}, \mathit{is}) \\
              &\mathsf{where}\ (S_o^1, S_i^1)\cup (S_o^2, S_i^2)\triangleq
                (S_o^1\cup S_o^2, S_i^1\cup S_i^2)\\
            % Product
            \interp { a \cdot b }(\mathit{os}, \mathit{is})
              \triangleq\ &
              \interp { a }(\mathit{os}, \mathit{is})\cap
              \interp { b }(\mathit{os}, \mathit{is}) \\
              &\mathsf{where}\ (S_o^1, S_i^1)\cap (S_o^2, S_i^2)\triangleq
                (S_o^1\cap S_o^2, S_i^1\cap S_i^2)\\
            % Negation
            \interp { \neg a }(\mathit{os}, \mathit{os})
              \triangleq\ &
              \mathsf{let}\ (S_o, S_i) = \interp {a}(\mathit{os}, \mathit{is}) \\
              &\mathsf{in}\ (\{\mathit{os}\} - S_o, \{\mathit{is}\} - S_i)
              \\
          \end{align*}
          \caption{The semantics of predicates in \sysname.  While predicates remain unchanged from \oldname, the theory changes slightly}
          \label{fig:garuda:sem:pol}
        \end{figure}

      \subsubsection{Semantics of Policies}\label{sec:spec:sem:pol}
        \begin{figure}
          \centering
          \begin{align*} 
            % Theory
            \interp{ \cdot }\ 
              :\ \ &
              \mathbf{Stream}(Eo)\times \mathbf{Stream}(Mi) \rightarrow \\
              & P(\mathbf{Stream}(Eo))\times P(\mathbf{Stream}(Mi)) 
              \\
            % Obfuscation
            \interp{(\lambda_{Encrypt}, \lambda_{Decrypt})}\
              \triangleq\
              & \mathsf{let}\ (os \rightarrow os' = \lambda_{Encrypt}),
                            \ (is \rightarrow is' = \lambda_{Decrypt})\\
              % & \mathsf{let}\ os \rightarrow os' = \lambda_{Encrypt}\\
              % & \mathsf{and}\ is \rightarrow is' = \lambda_{Decrypt}\\
              & \mathsf{in}\
              (\{\mathit{os'}\}, \{\mathit{is'}\})
              \\
            % Injection Action
            \interp { inj_{Eo}(o) }(\mathit{os}, \mathit{is})
              \triangleq\ &
              (\{\mathit{o : os}\}, \{\mathit{is}\}) 
              % or ::?  ; is sequential
              \\
            % Injection Result
            \interp { inj_{Mi}(i) }(\mathit{os}, \mathit{is})
              \triangleq\ &
              (\{\mathit{os}\},\{ \mathit{i : is}\})
              \\
            % Modification
            \interp { f \leftarrow n }(os, is)
              \triangleq\ &
              (\mathsf{map}\ (f\leftarrow n)\ \{os\},\
               \mathsf{map}\ (f\leftarrow n)\ \{is\})
              \\ %  could use editing?
            % Policy Union
            \interp { p + q }(\mathit{os}, \mathit{is})
              \triangleq\ &
              \interp { p }(\mathit{os}, \mathit{is})\cup
              \interp { q }(\mathit{os}, \mathit{is}) \\
              &\mathsf{where}\ (S_o^1, S_i^1)\cup (S_o^2, S_i^2)\triangleq
                (S_o^1\cup S_o^2, S_i^1\cup S_i^2)\\
              % A union of the streams gives a nondeterministic machine.
            % Policy Concatnation
            \interp { p \cdot q }(\mathit{os}, \mathit{is})
              \triangleq\ &
              \mathsf{let}\ (S_o, S_i) = \interp{p}(os, is)\\
              &\mathsf{in}\ \bigcup \{\interp{q}(\mathit{os}',\mathit{is}')\ |\ \mathit{os}'\in S_o, \mathit{is'}\in S_i\}\\
              \\
            % Defn. of Filter
            \interp{\mathsf{filter}\ f}(S)
              \triangleq\ & \{l \in S\ |\ f(l) = \mathsf{true}\}\\
            % Defn. of Map
            \interp{\mathsf{map}\ g}(S)
              \triangleq\ &
              \{ g(l)\ |\ l\in S \} 
          \end{align*}
          \caption{The semantics of policies in \sysname.}
          \label{fig:spec:sem:pol}
        \end{figure}
  %}
%%%%%%%%%%%%%%%%%%%%%%%%
%%% END OF SECTION 3 %%%
%%% GARUDA SEMANTICS %%%
%%%%%%%%%%%%%%%%%%%%%%%%


%%%%%%%%%%%%%%%%%%%%%%%
%%% BEGIN SECTION 5 %%%
%%% COMPILATION     %%%
%%%%%%%%%%%%%%%%%%%%%%%
  \clearpage
  \section{Compilation to Verilog}\label{sec:compile}
    \subsection{The Intermediate Language}\label{sec:compile:int}

      \begin{figure}
        \centering
        \begin{tabular}{l r r l l}
          %Definitions table
          Values        & $v$     & $::=$     & $\mathsf{INSTR\ |\ RES}$ &\\
          Registers     & $b$     & $::=$     & $\mathsf{reg}$           &\\
          \\
          %Predicates table
          Expressions & $e$ & $::=$  & $read(b)$       
                            & \textit{Read Reg}\\
                      &     & $\mid$ & $write(b,v)$    
                            & \textit{Write Value to Reg}\\
                      &     & $\mid$ & $let\ x = e_1\ in\ e_2$ 
                            & \textit{Assignment}\\  
                      &     & $\mid$ & $e_1\ ||\ e_2$ 
                            & \textit{Parallel}\\
                      &     & $\mid$ & $f(e_1)$        
                            & \textit{Apply Function} \\  
                      &     & $\mid$ & $if\ v = n\ then\ e_1\ else\ e_2$
                            & \textit{Conditional} \\
                      &     & $\mid$ & $e_1\ \&\&\ e_2$ 
                            & \textit{Product (AND)}\\
                      &     & $\mid$ & $e_1 ; e_2$
                            & \textit{Concatenation}\\
                      &     & $\mid$ & $forever\ e$ 
                            & \textit{Hardwire}
        \end{tabular}
        \caption{The language \sysname uses as a means to facilitate compilation.}
        \label{fig:compile:int}
      \end{figure}
    \subsection{Compiling \sysname to the Intermediate}\label{sec:compile:compile}
      We define a function $\mathbf{C}$ to be the compilation of some predicate or policy in \sysname to the intermediate language.
      $\mathbf{C}$ operates on the four ports of the \sysname monitor.
      These consist of the inputs and outputs of each of the instruction and result streams.
      Assume all $\mathbf{C}$ are short for $\mathbf{C}_{(i_{in},i_{out},r_{in},r_{out})}$ if unspecified.

      \subsubsection{Compiling Predicates}\label{sec:compile:compile:pred}
        \begin{figure}
          \begin{align*}
            %Falsity
            \mathbf{C}[0]\ 
              =\ &
              let\ i\_bog = \text{new buf}\\
              &\quad\ \ r\_bog = \text{new buf}\\
              &in\ write(i_{out}, i\_bog)\\
              &\quad  write(r_{out}, r\_bog)\\
            %Identity
            \mathbf{C}[1]\ 
              =\ &
              write(i_{out}, read(i_{in}))\\
              &write(r_{out}, read(r_{in}))
              \\
            %Filter
            \mathbf{C}[f = n]\
              =\
              &if\ i_{in}=n\ then\\
              &\quad write(i_{out}, read(i_{in}))\\
              &\quad else\ \mathbf{C}_{(i_{in}, i_{out}, -, -})[0]\\
              &if\ r_{in}=n\ then\\
              &\quad write(r_{out}, read(r_{in}))\\
              &\quad else\ \mathbf{C}_{(-, -, r_{in}, r_{out}})[0]\\
            %Predicate Addition
            \mathbf{C}[\mathsf{test}(a + b)]\ 
              =\ &
              let\ e_{aii}, e_{ari}, e_{bii}, e_{bri} = \text{new buf}\ in\\
              & let\ e_{aio}, e_{aro}, e_{bio}, e_{bro} = \text{new buf}\ in\\
                 &DeMux(i_{in}, e_{aii}, e_{bii})\ ||\ DeMux(r_{in}, e_{ari}, e_{bri})\\
              &let\ e_a = \mathbf{C}_{(e_{aii},e_{aio},e_{ari},e_{aro})}[a]\\
              &\quad\ \ e_b = \mathbf{C}_{(e_{bii},e_{bio},e_{bri},e_{bro})}[b]\\
              &in\ Mux(e_a\ \ e_b, (i_{out},r_{out}))\\
            %Predicate Product
            \mathbf{C}[\mathsf{test}(a \cdot b)]\ 
              =\ &\mathbf{C}[\mathsf{test}(a) \cdot \mathsf{test}(b)] \\
            %Predicate Negation
            \mathbf{C}[\neg a]\ 
              =\ 
              &if\ \neg a\ then\\
              &\quad write(i_{out}, read(i_{in}))\\
              &\quad write(r_{out}, read(r_{in}))\\
              &else\ \mathbf{C}[0]\\
          \end{align*}
          \caption{Compilation of \sysname predicates into the intermediate language.}
          \label{fig:compile:compile:pred}
        \end{figure}

      \subsubsection{Compiling Policies}\label{sec:compile:compile:pol}
        \begin{figure}
          \begin{align*}
            %Inject Instruction
            \mathbf{C}[inj_iV]\ 
              =\ &
              write(i_{out}, V)\\
            %Inject Result
            \mathbf{C}[inj_rV]\ 
              =\ &
              write(r_{out}, V)\\
            %Assignment
            \mathbf{C}[f \leftarrow n]\ 
              =\ &
              let\ i' = f(read(i_{in}))\\
              &\quad\ \ r' = f(read(r_{in}))\\
              &in\ write(i_{out}, i')\\
              &\quad \ write(r_{out}, r')\\
            %Policy Nondeterministic Choice
            \mathbf{C}[p + q]\ 
              =\ &
              let\ e_p = \mathbf{C}[p]\\
              &\quad\ \ e_q = \mathbf{C}[q]\\
              &in\ Mux(e_p, e_q, (i_{out},r_{out}))\\
            %Policy Sequence
            \mathbf{C}[p \cdot q]\ 
              =\ &
              let\ i_{mid} = \text{new buf}\\
              &\quad\ \ r_{mid} = \text{new buf}\\
              &\quad\ \ e_p = \mathbf{C}_{(i_{in}, i_{mid}, r_{in}, r_{mid})}[p]\\
              &\quad\ \ e_q = \mathbf{C}_{(i_{mid}, i_{out}, r_{mid}, r_{out})}[q]\\
              &in\ e_p\ ;\ e_q
          \end{align*}
          \caption{Compilation of \sysname policies into the intermediate language.}
          \label{fig:compile:compile:pol}
        \end{figure}


%%%%%%%%%%%%%%%%%%%%%%%
%%% BEGIN SECTION 5 %%%
%%% APPLICATIONS    %%%
%%%%%%%%%%%%%%%%%%%%%%%
  \clearpage
  \section{Applications of \sysname}

    %%%%%%%%%%%%%%%%%%%%%%%%%%%%
    %%% BEGIN SUBSECTION 5.1 %%%
    %%% STANDARD TAINT       %%%
    %%%%%%%%%%%%%%%%%%%%%%%%%%%%
    \subsection{Standard Taint}
    

      In the previous version of \oldname, taint was implemented by tagging the most significant bit of a register. 
      This caused no hardware overhead to maintain in the pipeline, but imposed an obvious bit-resolution hindrance.
      In \sysname, the definition of such a policy is identical, but the compilation is different.

      We assume a MIPS-like architecture for this example.
      The op codes of any given instruction are no more than 3; $\mathsf{IN_1}$, $\mathsf{IN_2}$, and $\mathsf{OUT}$
      We denote these as $\mathsf{RS}$, $\mathsf{RT}$, and $\mathsf{RD}$.

      Let's suppose you want your taint to propogate on the inputs of any arithmetic instructions.
      Additionally, you prohibit any tainted instructions or addresses from accessing memory.
      In \sysname, we can define this as follows.
      Please note that $\mathsf{ALU}$ can further be expanded to specific types of ALU instructions

      \[
      \begin{array}{r c l}
        %Define inputs
        \mathit{Instruction\ Fields}\ f_i &::=& \mathsf{Taint_{RS}}, \mathsf{Taint_{RT}}, \mathsf{Taint_{RD}}, \mathsf{OP}\\
        \mathsf{OP} &::=& \mathsf{MEM_{READ}}\ |\ \mathsf{MEM_{WRITE}}\ |\ \mathsf{ALU} |\ \cdots \\
        \mathit{Result\ Fields}\ f_r &::=& \mathsf{(*empty*)}\\\\

        %Op Types
        \mathsf{Arith}   &\triangleq& \mathsf{OP} = \mathsf{ALU}\\
        \mathsf{Read}    &\triangleq& \mathsf{OP} = \mathsf{MEM_{READ}}\\
        \mathsf{Write}   &\triangleq& \mathsf{OP} = \mathsf{MEM_{WRITE}}\\
        \mathsf{AnyMem}  &\triangleq& \mathsf{Read} + \mathsf{Write}\\\\

        %Sub Rules
        \mathsf{TaintedInstr} &\triangleq& ( \mathsf{Taint_{RS}} = \mathsf{TRUE} ) + ( \mathsf{Taint_{RT}} = \mathsf{TRUE} )\\
        \mathsf{TaintRes}     &\triangleq& \mathsf{Taint_{RD}} \leftarrow \mathsf{TRUE}\\\\

        %Rule
        \mathsf{PropTaintALU} &\triangleq&
          %% Formerly injected TaintRes, but update (\leftarrow) is a policy of its own that operates on fields
          act(\mathsf{Arith}\cdot \mathsf{TaintedInstr}\cdot \mathsf{TaintRes})\\
          &&+
          act(\mathsf{Arith}\cdot \neg \mathsf{TaintedInstr})\\
        \mathsf{NoTaintMem} &\triangleq&
          act(\mathsf{AnyMem}\cdot \neg \mathsf{TaintedInstr})\\
        \mathsf{SecureMem} &\triangleq& \mathsf{PropTaintALU} + \mathsf{NoTaintMem}\\
          &&+
          act(\neg (\mathsf{Arith} + \mathsf{AnyMem}))
      \end{array}
      \]\\
    %%%%%%%%%%%%%%%%%%%%%%%%%%
    %%% END SUBSECTION 5.1 %%%
    %%% STANDARD TAINT     %%%
    %%%%%%%%%%%%%%%%%%%%%%%%%%

    %%%%%%%%%%%%%%%%%%%%%%%%%%%%
    %%% BEGIN SUBSECTION 5.2 %%%
    %%% SPECULATIVE TAINT    %%%
    %%%%%%%%%%%%%%%%%%%%%%%%%%%%
      \subsection{Speculative Taint}

      \cite{2019stt}
      \[
      \begin{array}{r c l}
        %Define inputs
        \mathit{Instruction\ Fields}\ f_i &::=& \mathsf{Taint_{RS}}, \mathsf{Taint_{RT}}, \mathsf{Taint_{RD}}, \mathsf{OP}\\
        \mathsf{OP} &::=& \mathsf{MEM_{READ}}\ |\ \mathsf{MEM_{WRITE}}\ |\ \mathsf{ALU} |\ \cdots \\
        \mathit{Result\ Fields}\ f_r &::=& \mathsf{(*empty*)}\\\\

        %Op Types
        \mathsf{Arith}   &\triangleq& \mathsf{OP} = \mathsf{ALU}\\
        \mathsf{Read}    &\triangleq& \mathsf{OP} = \mathsf{MEM_{READ}}\\
        \mathsf{Write}   &\triangleq& \mathsf{OP} = \mathsf{MEM_{WRITE}}\\
        \mathsf{AnyMem}  &\triangleq& \mathsf{Read} + \mathsf{Write}\\\\

        %Sub Rules
        \mathsf{TaintedInstr} &\triangleq& ( \mathsf{Taint_{RS}} = \mathsf{TRUE} ) + ( \mathsf{Taint_{RT}} = \mathsf{TRUE} )\\
        \mathsf{TaintRes}     &\triangleq& \mathsf{Taint_{RD}} \leftarrow \mathsf{TRUE}\\\\

        %Rule
        \mathsf{PropTaintALU} &\triangleq&
          %% Formerly injected TaintRes, but update (\leftarrow) is a policy of its own that operates on fields
          act(\mathsf{Arith}\cdot \mathsf{TaintedInstr}\cdot \mathsf{TaintRes})\\
          &&+
          act(\mathsf{Arith}\cdot \neg \mathsf{TaintedInstr})\\
        \mathsf{NoTaintMem} &\triangleq&
          act(\mathsf{AnyMem}\cdot \neg \mathsf{TaintedInstr})\\
        \mathsf{SecureMem} &\triangleq& \mathsf{PropTaintALU} + \mathsf{NoTaintMem}\\
          &&+
          act(\neg (\mathsf{Arith} + \mathsf{AnyMem}))
      \end{array}
      \]\\
    %%%%%%%%%%%%%%%%%%%%%%%%%%
    %%% END SUBSECTION 5.2 %%%
    %%% SPECULATIVE TAINT  %%%
    %%%%%%%%%%%%%%%%%%%%%%%%%%


%%%%%%%%%%%%%%%%%
%%% CITATIONS %%%
%%%%%%%%%%%%%%%%%
  \clearpage
  % \begin{singlespace}
    % \bibliographystyle{ACM-Reference-Format}
    \bibliographystyle{unsrt}
    \bibliography{Language}
  % \end{singlespace}

\end{document}
